\subsection{Symmetric Monoidal Functor}
\label{subsec:symmetric_monoidal_functor}
\begin{definition}
    A symmetric monoidal functor between SMCs $(\mathbb{C},\bullet,1,\alpha,\lambda,\rho,\gamma)$ and 
    $(\mathbb{C}',\bullet',1',\alpha',\lambda',\rho',\gamma')$ is a functor $F:\mathbb{C} \rightarrow \mathbb{C}'$
    equipped with
    \begin{enumerate}
        \item A morphism $m_{1'}: 1' \rightarrow F1$.
        \item For any two objects A and B in $\mathbb{C}$, a natural transformation $m_{A,B}:F(A) \bullet 'F(B) \rightarrow F(A \bullet B)$
    \end{enumerate}
    These must satisfy the following diagrams:
    \begin{center}
        \begin{math}
            \begin{array}{c}
                \begin{array}{lr}
                    \bfig
                    \square|alrb|/->`<-`->`->/<700,500>[
                        F1 \bullet ' FA`
                        F(1 \bullet A)`
                        1' \bullet ' FA`
                        FA;
                        m_{1,A}`
                        m_1 \bullet ' id_{FA}`
                        F(\lambda_A)`
                        \lambda_{FA}'
                    ]
                    \efig
                    &
                    \bfig
                    \square|alrb|/->`<-`->`->/<700,500>[
                    FA \bullet ' F1`
                    F (A \bullet 1)`
                    FA \bullet ' 1'`
                    FA;
                    m_{A,1}`
                    id_{FA} \bullet ' m_{1'}`
                    F(\rho_A)`
                    \rho_{FA}'
                    ]
                    \efig
                \end{array}
                \\\\
                \bfig
                \square|alrb|/->`<-``->/<1200,500>[
                    (FA \bullet ' FB) \bullet ' FC`
                    F(A \bullet B) \bullet ' FC`
                    FA \bullet ' (FB \bullet ' FC)`
                    FA \bullet ' F(B \bullet C);
                    m_{A,B} \bullet ' id_{FC}`
                    \alpha_{FA,FB,FC}'`
                    `
                    id_{FA} \bullet ' m_{B,C}
                ]
                \square(1200,0)|alrb|/->``<-`->/<1200,500>[
                    F(A \bullet B) \bullet ' FC`
                    F((A \bullet B)\bullet C)`
                    FA \bullet ' F(B \bullet C)`
                    F(A \bullet (B \bullet C));
                    m_{A \bullet B,C}`
                    `
                    F(\alpha_{A,B,C})`
                    m_{A,B \bullet C}
                ]
                \efig
                \\\\
                \bfig
                \square|alrb|<700,500>[
                    FA \bullet ' FB`
                    F(A \bullet B)`
                    FB \bullet ' FA`
                    F(B \bullet A);
                    m_{A,B}`
                    \gamma_{A,B}'`
                    F(\gamma_{A,B})`
                    m_{B,A}
                ]
                \efig
            \end{array}
        \end{math}
    \end{center}
    However in this particular case, assuming that ! is a symmetric monoidal (endo) functor means that ! comes equipped
    with a natural transformation
    \begin{center}
        \begin{math}
            m_{A,B}:!A \otimes !B \rightarrow !(A \otimes B)
        \end{math}
    \end{center}
    and a morphism
    \begin{center}
        \begin{math}
            m_I : I \rightarrow !I
        \end{math}
    \end{center}
    (where $m_I$ is just the nullary version of the natural transformation.)  The diagrmas given in the above definition become
    the following:
    \begin{center}
        \begin{math}
            \begin{array}{c}
                \begin{array}{lr}
                    \bfig
                    \square|alrb|/->`<-`->`->/<600,500>[
                        !I \otimes !A`
                        !(I \otimes A)`
                        I \otimes !A`
                        !A;
                        m_{I,A}`
                        m_{I} \otimes id_{!A}`
                        !(\lambda_A)`
                        \lambda_{!A}
                    ]
                    \efig
                    &
                    \bfig
                    \square|alrb|/->`<-`->`->/<600,500>[
                        !A \otimes !I`
                        !(A \otimes I)`
                        !A \otimes I`
                        !A;
                        m_{A,I}`
                        id_{!A} \otimes m_I`
                        !(\rho_A)`
                        \rho_{!A}
                    ]
                    \efig
                \end{array}
                \\\\
                \bfig
                \square|alrb|/->`<-``->/<1200,500>[
                    (!A \otimes !B) \otimes !C`
                    !(A \otimes  B) \otimes !C`
                    !A \otimes (!B \otimes !C)`
                    !A \otimes !(B \otimes  C);
                    m_{A,B} \otimes id_{!C}`
                    \alpha_{!A,!B,!C}`
                    `
                    id_{!A} \otimes m_{B,C}
                ]
                \square(1200,0)|alrb|/->``<-`->/<1200,500>[
                    !(A \otimes  B) \otimes !C`
                    !((A \otimes B)\otimes C)`
                    !A \otimes !(B \otimes  C)`
                    !(A \otimes (B \otimes C));
                    m_{A \otimes B,C}`
                    `
                    !(\alpha_{A,B,C})`
                    m_{A,B \otimes C}
                ]
                \efig
                \\\\
                \bfig
                \square|alrb|<700,500>[
                    !A \otimes !B`
                    !(A \otimes B)`
                    !B \otimes !A`
                    !(B \otimes A);
                    m_{A,B}`
                    \gamma_{A,B}'`
                    !(\gamma_{A,B})`
                    m_{B,A}
                ]
                \efig
            \end{array}
        \end{math}
    \end{center}
\end{definition}
\subsection{Symmetric Monoidal Functor (Strict and Strong)}
\label{subsec:symmetric_monoidal_functor_s_and_s}
\begin{definition}
    A symmetric monoidal functor, $(F,m_{A,B},m_{1'}): \mathbb{C} \rightarrow \mathbb{C}'$, is said to be
    \begin{enumerate}
        \item Strict if $m_{A,B}$ and $m_{1'}$ are identities.
        \item Strong if $m_{A,B}$ and $m_{1'}$ are natural isomorphisms.
    \end{enumerate}
    The equation in context for Dereliction gives us that
    \begin{center}
        \begin{math}
            \begin{array}{c}
                \bfig
                \morphism(0,500)|a|<500,0>[!A_1`!A_1;id_{!A_1}]
                \qtriangle(500,0)|alr|<500,500>[
                    !A_1`
                    !B`
                    B;
                    (\epsilon_A;f)^*`
                    \epsilon_A;f`
                    \epsilon_B
                ]
                \efig
            \end{array}
        \end{math}
    \end{center}
    commutes, or in other words
    \begin{center}
        \begin{math}
            \begin{array}{c}
                \bfig
                \square[!A`!B`A`B;!f`\epsilon_A`\epsilon_B`f]
                \efig
            \end{array}
        \end{math}
    \end{center}
    commutes.  Given that we have made the assumption that ! is a (symmetric) monoidal functor, this diagram suggests
    that $\epsilon$ is a monoidal natural transformation.  We shall make this assumption and write $\epsilon$ for the monoidal 
    natural transformation $\epsilon:! \rightarrow Id$.  
\end{definition}