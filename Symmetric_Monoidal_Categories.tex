\begin{definition}
    \label{def:symmetric_monoidal_category}
    A symmetric monoidal catergory (SMC), $(\mathbb{C},\otimes,I,\alpha,\lambda,\rho,\gamma)$, is a category $\mathbb{C}$ 
    equippped with a bifunctor $\otimes:\mathbb{C}\times\mathbb{C}\rightarrow\mathbb{C}$ with a neutral element I and natural isomorphisms 
    $\alpha, \lambda, \rho,$ and $\gamma$:
    \begin{enumerate}
        \item $\alpha_{A,B,C} : A \otimes (B \otimes C) \xrightarrow{\sim} (A \otimes B) \otimes C$
        \item $\lambda_A : I \otimes A \xrightarrow{\sim} A$
        \item $\rho_A : A \otimes I \xrightarrow{\sim} A$
        \item $\gamma_{A,B} : A \otimes B \xrightarrow{\sim} B \otimes A$
    \end{enumerate}
    which make the following 'coherence' diagrams commute.
    \begin{center}
        \begin{math}
            \begin{array}{c}
                \bfig
                \ptriangle|all|/->`->`/<1500,500>[
                    A \otimes (B \otimes (C \otimes D))`
                    (A \otimes B) \otimes (C \otimes D)`
                    A \otimes ((B \otimes C)\otimes D);
                    \alpha_{A,B,C \otimes D}`
                    id_A \otimes \alpha_{B,C,D}`
                ]
                \qtriangle(1500,0)|alr|/->``<-/<1500,500>[
                    (A \otimes B) \otimes (C \otimes D)`
                    ((A \otimes B)\otimes C)\otimes D`
                    (A \otimes (B \otimes C))\otimes D;
                    \alpha_{A \otimes B,C,C}`
                    `
                    \alpha_{A,B,C} \otimes id_D
                    ]
                \morphism|b|/->/<3000,0>[
                    A \otimes ((B \otimes C)\otimes D)`
                    (A \otimes(B \otimes C))\otimes D;
                    \alpha_{A,B \otimes C,D}
                    ]
                \efig
                \\\\
                \bfig
                \square|alrb|/->`->``->/<1000,500>[
                    (A \otimes B) \otimes C`
                    A \otimes (B \otimes C)`
                    (B \otimes A) \otimes C`
                    B \otimes (A \otimes C);
                    \alpha_{A,B,C}`
                    \gamma_{A,B} \otimes id_C`
                    `
                    \alpha_{B,A,C}
                ]
                \square(1000,0)|arlb|/->``->`->/<1000,500>[
                    A \otimes (B \otimes C)`
                    (B \otimes C) \otimes A`
                    B \otimes (A \otimes C)`
                    B \otimes (C \otimes A);
                    \gamma_{A,B \otimes C}`
                    `
                    \alpha_{B,C,A}`
                    id_B \otimes \gamma_{A,C}
                ]
                \efig
                \\\\
                \begin{array}{lr}
                    \bfig
                    \square|alrr|/->`->`->`=/<1000,500>[
                        A \otimes (I \otimes B)`
                        (A \otimes I) \otimes B`
                        A \otimes B`
                        A \otimes B;
                        \alpha_{A,I,B}`
                        id_A \otimes \lambda_B`
                        \rho_A \otimes id_B`]
                    \efig
                    &
                    \bfig
                    \qtriangle|abr|/->`=`->/[
                        A \otimes B`
                        B \otimes A`
                        A \otimes B;
                        \gamma_{A,B}`
                        `
                        \gamma_{B,A}]
                    \efig                    
                \end{array}\\\\
                \bfig
                \square|alrb|/->`->`->`=/[
                    A \otimes I`
                    I \otimes A`
                    A`
                    A;
                    \gamma_{A,I}`
                    \rho_A`
                    \lambda_A`
                    ]
                \efig
            \end{array}
        \end{math}
    \end{center}
    The following equality is also require to hold:
    \begin{center}
        \begin{math}
            \begin{array}{lcl}
                \lambda_I & = & \rho_I : I \otimes I \rightarrow I
            \end{array}
        \end{math}
    \end{center}
    \cite{bierman1993}
\end{definition}



\begin{definition}
\label{def:symmetric_monoidal_closded_category}
    A symmetric monoidal closed category (SMCC), $(\mathbb{C},\otimes,\multimap,I,\alpha,\lambda,\rho,\gamma)$,
    is a SMC such that for all objects A in $\mathbb{C}$, the functor $-\otimes A$ has a specified right adjoint $A\multimap -$.
\end{definition}


\begin{definition}
\label{def:symmetric_monoidal_functor}
    A symmetric monoidal functor between SMCs $(\mathbb{C},\otimes,I,\alpha,\lambda,\rho,\gamma)$ and 
    $(\mathbb{C}',\otimes',I',\alpha',\lambda',\rho',\gamma')$ is a functor $F:\mathbb{C} \rightarrow \mathbb{C}'$
    equipped with
    \begin{enumerate}
        \item A morphism $m_{I'}: I' \rightarrow FI$.
        \item For any two objects A and B in $\mathbb{C}$, a natural transformation $m_{A,B}:F(A) \otimes 'F(B) \rightarrow F(A \otimes B)$
    \end{enumerate}
    These must satisfy the following diagrams:
    \begin{center}
        \begin{math}
            \begin{array}{c}
                \begin{array}{lr}
                    \bfig
                    \square|alrb|/->`<-`->`->/<700,500>[
                        FI \otimes ' FA`
                        F(I \otimes A)`
                        I' \otimes ' FA`
                        FA;
                        m_{I,A}`
                        m_I \otimes ' id_{FA}`
                        F(\lambda_A)`
                        \lambda_{FA}'
                    ]
                    \efig
                    &
                    \bfig
                    \square|alrb|/->`<-`->`->/<700,500>[
                    FA \otimes ' FI`
                    F (A \otimes I)`
                    FA \otimes ' I'`
                    FA;
                    m_{A,I}`
                    id_{FA} \otimes ' m_{I'}`
                    F(\rho_A)`
                    \rho_{FA}'
                    ]
                    \efig
                \end{array}
                \\\\
                \bfig
                \square|alrb|/->`<-``->/<1200,500>[
                    (FA \otimes ' FB) \otimes ' FC`
                    F(A \otimes B) \otimes ' FC`
                    FA \otimes ' (FB \otimes ' FC)`
                    FA \otimes ' F(B \otimes C);
                    m_{A,B} \otimes ' id_{FC}`
                    \alpha_{FA,FB,FC}'`
                    `
                    id_{FA} \otimes ' m_{B,C}
                ]
                \square(1200,0)|alrb|/->``<-`->/<1200,500>[
                    F(A \otimes B) \otimes ' FC`
                    F((A \otimes B)\otimes C)`
                    FA \otimes ' F(B \otimes C)`
                    F(A \otimes (B \otimes C));
                    m_{A \otimes B,C}`
                    `
                    F(\alpha_{A,B,C})`
                    m_{A,B \otimes C}
                ]
                \efig
                \\\\
                \bfig
                \square|alrb|<700,500>[
                    FA \otimes ' FB`
                    F(A \otimes B)`
                    FB \otimes ' FA`
                    F(B \otimes A);
                    m_{A,B}`
                    \gamma_{A,B}'`
                    F(\gamma_{A,B})`
                    m_{B,A}
                ]
                \efig
            \end{array}
        \end{math}
    \end{center}
    However in this particular case, assuming that ! is a symmetric monoidal (endo) functor means that ! comes equipped
    with a natural transformation
    \begin{center}
        \begin{math}
            m_{A,B}:!A \otimes !B \rightarrow !(A \otimes B)
        \end{math}
    \end{center}
    and a morphism
    \begin{center}
        \begin{math}
            m_I : I \rightarrow !I
        \end{math}
    \end{center}
    (where $m_I$ is just the nullary version of the natural transformation.)  The diagrmas given in the above definition become
    the following:
    \begin{center}
        \begin{math}
            \begin{array}{c}
                \begin{array}{lr}
                    \bfig
                    \square|alrb|/->`<-`->`->/<600,500>[
                        !I \otimes !A`
                        !(I \otimes A)`
                        I \otimes !A`
                        !A;
                        m_{I,A}`
                        m_{I} \otimes id_{!A}`
                        !(\lambda_A)`
                        \lambda_{!A}
                    ]
                    \efig
                    &
                    \bfig
                    \square|alrb|/->`<-`->`->/<600,500>[
                        !A \otimes !I`
                        !(A \otimes I)`
                        !A \otimes I`
                        !A;
                        m_{A,I}`
                        id_{!A} \otimes m_I`
                        !(\rho_A)`
                        \rho_{!A}
                    ]
                    \efig
                \end{array}
                \\\\
                \bfig
                \square|alrb|/->`<-``->/<1200,500>[
                    (!A \otimes !B) \otimes !C`
                    !(A \otimes  B) \otimes !C`
                    !A \otimes (!B \otimes !C)`
                    !A \otimes !(B \otimes  C);
                    m_{A,B} \otimes id_{!C}`
                    \alpha_{!A,!B,!C}`
                    `
                    id_{!A} \otimes m_{B,C}
                ]
                \square(1200,0)|alrb|/->``<-`->/<1200,500>[
                    !(A \otimes  B) \otimes !C`
                    !((A \otimes B)\otimes C)`
                    !A \otimes !(B \otimes  C)`
                    !(A \otimes (B \otimes C));
                    m_{A \otimes B,C}`
                    `
                    !(\alpha_{A,B,C})`
                    m_{A,B \otimes C}
                ]
                \efig
                \\\\
                \bfig
                \square|alrb|<700,500>[
                    !A \otimes !B`
                    !(A \otimes B)`
                    !B \otimes !A`
                    !(B \otimes A);
                    m_{A,B}`
                    \gamma_{A,B}'`
                    !(\gamma_{A,B})`
                    m_{B,A}
                ]
                \efig
            \end{array}
        \end{math}
    \end{center}
    \cite{bierman1993}
\end{definition}

\begin{definition}
    A symmetric monoidal functor, $(F,m_{A,B},m_{I'}): \mathbb{C} \rightarrow \mathbb{C}'$, is said to be
    \begin{enumerate}
        \item Strict if $m_{A,B}$ and $m_{I'}$ are identities.
        \item Strong if $m_{A,B}$ and $m_{I'}$ are natural isomorphisms.
    \end{enumerate}
    \cite{bierman1993}
\end{definition}