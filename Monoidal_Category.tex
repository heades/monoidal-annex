\begin{definition}
    A monoidal category $(\mathbb{C},\otimes,I,\alpha,\lambda,\rho)$ is a category $\mathbb{C}$, a bifunctor $\otimes : \mathbb{C} \times \mathbb{C} \rightarrow \mathbb{C}$, 
    an object $I \in \mathbb{C}$, and three natural isomorphisms $\alpha,\lambda,\rho$.  Where,
    \begin{center}
        \begin{math}
            \alpha = \alpha_{A,B,C}:A \otimes(B \otimes C) \cong (A \otimes B) \otimes C
        \end{math}
    \end{center}
    is natural for all $A,B,C \in \mathbb{C}$ and the diagram
    \begin{center}
        \begin{math}
            \bfig
            \ptriangle|all|/->`->`/<1500,500>[
                A \otimes (B \otimes (C \otimes D))`
                (A \otimes B) \otimes (C \otimes D)`
                A \otimes ((B \otimes C)\otimes D);
                \alpha`
                1 \otimes \alpha`
            ]
            \qtriangle(1500,0)|alr|/->``<-/<1500,500>[
                (A \otimes B) \otimes (C \otimes D)`
                ((A \otimes B)\otimes C)\otimes D`
                (A \otimes (B \otimes C))\otimes D;
                \alpha`
                `
                \alpha \otimes 1
                ]
            \morphism|b|/->/<3000,0>[
                A \otimes ((B \otimes C)\otimes D)`
                (A \otimes(B \otimes C))\otimes D;
                \alpha
                ]
            \efig
        \end{math}
    \end{center}
    commutes for all $A,B,C,D \in \mathbb{C}$.  $\gamma$ and $\rho$ are natural
    \begin{center}
        \begin{math}
            \begin{array}{lr}
                \gamma_A : I \otimes A \cong A & \rho_A : A \otimes I \cong A
            \end{array}
        \end{math}
    \end{center} 
    for all objects $A \in \mathbb{C}$, the diagram
    \begin{center}
        \begin{math}
            \bfig
                    \square|alrr|/->`->`->`=/<1000,500>[
                        A \otimes (I \otimes C)`
                        (A \otimes I) \otimes C`
                        A \otimes C`
                        A \otimes C;
                        \alpha`
                        1 \otimes \lambda`
                        \rho_A \otimes 1`]
                    \efig
        \end{math}
    \end{center}
    \cite{maclane1971}
\end{definition}


\begin{definition}
\label{def:monoidal_functor}
    A monoidal functor or lax monoidal functor $(F,m)$ between monoidal categories $(\mathbb{C}, \otimes, I)$ and
    $(\mathbb{D}, \otimes ', I')$ is a functor $F:\mathbb{C} \rightarrow \mathbb{D}$ equipped with nathral transformations
    \begin{center}
        \begin{math}
            \begin{array}{lr}
                m_{A,B}' : FA \otimes ' FB \rightarrow F(A \otimes B) & m'' : I' \rightarrow FI
            \end{array}
        \end{math}
    \end{center}
    where the following diagrams commute in the category $\mathbb{D}$ for all objects $A,B,C \in \mathbb{C}$
    \begin{center}
        \begin{math}
            \begin{array}{c}
                \bfig
                    \square|alrr|/->`->`->`/<1500,500>[
                        (FA \otimes ' FB) \otimes ' FC`
                        FA \otimes ' (FB \otimes ' FC)`
                        F(A \otimes B) \otimes ' FC`
                        FA \otimes ' F(B \otimes C);
                        \alpha '`
                        m \otimes FC`
                        FA \otimes m`]
                    \square(0,-500)|alrb|/`->`->`->/<1500,500>[
                        F(A \otimes B) \otimes ' FC`
                        FA \otimes ' F(B \otimes C)`
                        F((A \otimes B)\otimes C)`
                        F(A \otimes (B \otimes C));
                        `
                        m`
                        m`
                        F \alpha
                    ]
                    
                \efig
                \\\\
                \begin{array}{lr}
                    \bfig
                        \square|alra|/->`->`<-`->/<1000,500>[
                            FA \otimes ' I'`
                            FA`
                            FA \otimes ' FI`
                            F(A \otimes I);
                            \rho'`
                            FA \otimes ' m`
                            F\rho`
                            m
                        ]
                    \efig
                    &
                    \bfig
                        \square|alra|/->`->`<-`->/<1000,500>[
                            I' \otimes ' FB`
                            FB`
                            FI \otimes FB`
                            F(I \otimes B);
                            \gamma '`
                            m \otimes ' FB`
                            F\gamma`
                            m
                        ]
                    \efig
                \end{array}
            \end{array}
        \end{math}
    \end{center}
    \cite{mellies2009}
\end{definition}


\begin{definition}
\label{def:oplax_monoidal_functor}
    An oplax(colax/comonoidal) monoidal functor $(F,n)$ between monoidal categories $(\mathbb{C}, \otimes, I)$ and
    $(\mathbb{D}, \otimes ', I')$ is a functor $F:\mathbb{C} \rightarrow \mathbb{D}$ and natural transformations
    \begin{center}
        \begin{math}
            \begin{array}{lr}
                n_{A,B}' : F(A \otimes B) \rightarrow FA \otimes ' FB & n'' : FI \rightarrow I'
            \end{array}
        \end{math}
    \end{center}
    where the following diagrams commute in the category $\mathbb{D}$, for all objects $A,B,C \in \mathbb{C}$
    \begin{center}
        \begin{math}
            \begin{array}{c}
                \bfig
                \square|alrr|/->`->`->`/<1500,500>[
                    F((A \otimes B)\otimes C)`
                    F(A \otimes (B \otimes C))`
                    F(A \otimes B) \otimes ' FC`
                    FA \otimes ' F(B \otimes C);
                    F \alpha`
                    n`
                    n`
                ]
                \square(0,-500)|alrb|/`->`->`->/<1500,500>[
                    F(A \otimes B) \otimes ' FC`
                    FA \otimes ' F(B \otimes C)`
                    (FA \otimes ' FB) \otimes ' FC`
                    FA \otimes ' (FB \otimes ' FC);
                    `
                    n \otimes ' FC`
                    FA \otimes ' n`
                    \alpha '
                ]
                \efig
                \\\\
                \begin{array}{lr}
                    \bfig
                        \square|alra|/->`->`<-`->/<1000,500>[
                            F(A \otimes I)`
                            FA`
                            FA \otimes ' FI`
                            FA \otimes ' I';
                            F\rho`
                            n`
                            \rho '`
                            FA \otimes ' n
                        ]
                    \efig
                    &
                    \bfig
                        \square|alra|/->`->`<-`->/<1000,500>[
                            F(I \otimes B)`
                            FB`
                            FI \otimes ' FB`
                            I' \otimes ' FB;
                            F\gamma`
                            n`
                            \gamma '`
                            n \otimes ' FB
                        ]
                    \efig
                \end{array}
            \end{array}
        \end{math}
    \end{center}
    \cite{mellies2009}
\end{definition}



\begin{definition}
\label{def:monoidal_natural_transformation}
$(F,m)$ and $(G,n)$ are (lax) monoidal functors between the monoidal categories:
\begin{center}
    \begin{math}
        (\mathbb{C}, \otimes, I) \rightarrow (\mathbb{D}, \otimes ', I')
    \end{math}
\end{center}
A monoidal natural transformation
\begin{center}
    \begin{math}
        \theta : (F,m) \Rightarrow (G,n) : (\mathbb{C}, \otimes, I) \rightarrow (\mathbb{D}, \otimes ', I')
    \end{math}
\end{center}
between the monoidal functors $(F,m)$ and $(G,n)$ is a naural transformation
\begin{center}
    \begin{math}
        \theta : F \Rightarrow G : \mathbb{C} \rightarrow \mathbb{D}
    \end{math}
\end{center}
between the underlying functors, where the following diamgrams commute, for all objects $A,B \in \mathbb{C}$
\begin{center}
    \begin{math}
        \begin{array}{lr}
            \bfig
                \square|alrb|<1000,500>[
                    FA \otimes ' FB`
                    GA \otimes ' GB`
                    F(A \otimes B)`
                    G(A \otimes B);
                    \theta_A \otimes ' \theta_B`
                    m`
                    n`
                    \theta_{A \otimes B}
                ]
            \efig
            &
            \bfig
                \Atriangle<500,600>[
                    I'`
                    FI`
                    GI;
                    m`
                    n`
                    \theta_I
                ]
            \efig
        \end{array}
    \end{math}
\end{center}
\cite{mellies2009}
\end{definition}


\begin{definition}
\label{def:oplax_monoidal_natural_transformation}
    A monoidal natural transformation
    \begin{center}
        \begin{math}
            \theta : (F,m) \Rightarrow (G,n) : (\mathbb{C}, \otimes, I) \rightarrow (\mathbb{D}, \otimes ', I')
        \end{math}
    \end{center}
    between two oplax(colax/comonoidal) monoidal functors $(F,m)$ and $(G,n)$ is a natural transformation
    \begin{center}
        \begin{math}
            \theta : F \Rightarrow G : \mathbb{C} \rightarrow \mathbb{D}
        \end{math}
    \end{center}
    between the underlying functors, where the following diagrmas commute, for all objects $A,B \in \mathbb{C}$
    \begin{center}
        \begin{math}
            \begin{array}{rl}
                \bfig
                    \square<1000,500>[
                        F(A \otimes B)`
                        G(A \otimes B)`
                        FA \otimes ' FB`
                        GA \otimes ' GB;
                        \theta_{A \otimes B}`
                        m`
                        n`
                        \theta_A \otimes ' \theta_B
                    ]
                \efig
                &
                \bfig
                    \Vtriangle<500,600>[
                        FI`
                        GI`
                        I';
                        \theta_I`
                        m`
                        n
                    ]
                \efig
            \end{array}
        \end{math}
    \end{center}
    \cite{mellies2009}
\end{definition}


\begin{definition}
\label{def:biclosed_monoidal_category}
    A monoidal category $\mathbb{C}$ is said to be biclosed if every $- \otimes Y$ has a right adjoint $[Y,-]$ and every 
    $X \otimes -$ has a right adjoint $\interp{X,-}$ \cite{kelly1982}
\end{definition}


\begin{definition}
\label{def:monoidal_adjunction}
    Given a pair of (lax) monoidal functors:
    \begin{center}
        \begin{math}
            \begin{array}{lr}
                (F_*,m) : (\mathbb{C}, \otimes, I) \rightarrow (\mathbb{D}, \otimes ' , I')
                &
                (F^*, n) : (\mathbb{D}, \otimes ', I') \rightarrow (\mathbb{C},\otimes,I).
            \end{array}
        \end{math}
    \end{center}
    A monoidal adjunction
    \begin{center}
        \begin{math}
            (F_*,m) \dashv (F^*, n)
        \end{math}
    \end{center}
    between the monoidal functors is defined as an adjunction $(F_*,F^*,\eta,\epsilon)$ between the underlying functors
    \begin{center}
        \begin{math}
            \begin{array}{lr}
                F_* : \mathbb{C} \rightarrow \mathbb{D}
                &
                F^* : \mathbb{D} \rightarrow \mathbb{C}
            \end{array}
        \end{math}
    \end{center}
    whose natural transformations 
    \begin{center}
        \begin{math}
            \begin{array}{lr}
                \eta : id_C \Rightarrow F^* \circ F_* & \epsilon : F_* \circ F^* \Rightarrow id_D
            \end{array}
        \end{math}
    \end{center}
    are monoidal \cite{mellies2009}
\end{definition}
