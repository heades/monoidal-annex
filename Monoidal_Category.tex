\subsection{Symmetric Monoidal Category}
\label{subsec:symmetric_monoidal_category}
\begin{definition}
    A symmetric monoidal catergory (SMC), $(\mathbb{C},\bullet,1,\alpha,\lambda,\rho,\gamma)$, is a category $\mathbb{C}$ 
    equippped with a bifunctor $\bullet:\mathbb{C}\times\mathbb{C}\rightarrow\mathbb{C}$ with a neutral element 1 and natural isomorphisms 
    $\alpha, \lambda, \rho,$ and $\gamma$:
    \begin{enumerate}
        \item $\alpha_{A,B,C} : A \bullet (B \bullet C) \xrightarrow{\sim} (A \bullet B) \bullet C$
        \item $\lambda_A : 1 \bullet A \xrightarrow{\sim} A$
        \item $\rho_A : A \bullet 1 \xrightarrow{\sim} A$
        \item $\gamma_{A,B} : A \bullet B \xrightarrow{\sim} B \bullet A$
    \end{enumerate}
    which make the following 'coherence' diagrams commute.
    \begin{center}
        \begin{math}
            \begin{array}{c}
                diagrams here
            \end{array}
        \end{math}
    \end{center}
    The following equality is also require to hold:
    \begin{center}
        \begin{math}
            \begin{array}{lcl}
                \lambda_1 & = & \rho_1 : 1 \bullet 1 \rightarrow 1
            \end{array}
        \end{math}
    \end{center}
\end{definition}

\subsection{Symmetric Monoidal Closed Category}
\label{subsec:symmetric_monoidal_closded_category}
\begin{definition}
    A symmetric monoidal closed category (SMCC), $(\mathbb{C},\bullet,\multimap,1,\alpha,\lambda,\rho,\gamma)$,
    is a SMC such that for all objects A in $\mathbb{C}$, the functor $-\otimes A$ has a specified right adjoint $A\multimap -$.
    \\
    \\
    \indent Let $\mathbb{C}$ be a SMC $(\mathbb{C},\bullet,1,\alpha,\lambda,\rho,\gamma)$. A structure , $M$ , in $\mathbb{C}$ for a given signature 
    $S_g$ is specified by giving an object $\interp{\sigma}$ in $\mathbb{C}$ for each type $\sigma$, and a morphism 
    $\interp{f}:\interp{\sigma_1}\bullet...\bullet \interp{\sigma_n} \rightarrow \interp{\tau}$ in $\mathbb{C}$ for each function symbol
    $f:\sigma_1 , ... , \sigma_n \rightarrow \tau$.  In the case where $n = 0$ then the structure assigns a morphism 
    $\interp{c}:1 \rightarrow \interp{\tau}$ to a constant $c:\tau$.
    \\
    \indent Given a context $\Gamma = [x_1:\sigma_1,...,x_n:\sigma_n]$ we define $\interp{\Gamma}$ to be the product
    $\interp{\sigma_1}\bullet ... \bullet \interp{\sigma_n}$.  We represent the empyt context with the neutral element 1.  We need
    to define the bracketing convention.  It shall be assumed that the tensor product is left associative, i.e. $A_1 \bullet A_2\bullet ... \bullet A_n$ 
    will be taken to mean $(...(A_1 \bullet A_2)\bullet...)\bullet A_n$.  We find it useful to define two 'book-keeping' functions,
    \begin{center}
        \begin{math}
            \begin{array}{l}
                Split(\Gamma,\Delta):\interp{\Gamma,\Delta} \rightarrow \interp{\Gamma} \bullet \interp{\Delta}\\\\
                Split(\Gamma,\Delta) \myeq
                \begin{cases}
                    \lambda^{-1}_{\Delta} & \text{If }\Gamma \text{ empty}\\
                    \rho^{-1}_{\Gamma} & \text{If } \Delta \text{ empty}\\
                    id_{\Gamma \bullet A} & \text{If } \Delta = A\\
                    Split(\Gamma,\Delta')\bullet id_{A};\alpha^{-1}_{\Gamma,\Delta',A} & \text{If } \Delta = \Delta',A
                \end{cases}
                \\\\
                Join(\Gamma,\Delta): \interp{\Gamma,\Delta} \rightarrow \interp{\Gamma} \bullet \interp{\Delta}\\\\
                Join(\Gamma,\Delta) \myeq 
                \begin{cases}
                    \lambda_\Delta & \text{If } \Gamma \text{ empty}\\
                    \rho_\Gamma & \text{If } \Delta \text{ empty}\\
                    id_{\Gamma \bullet A} & \text{If }\Delta = A\\
                    \alpha_{\Gamma,\Delta',A};Join(\Gamma,\Delta')\bullet id_A & \text{If } \Delta = \Delta', A
                \end{cases}
            \end{array}
        \end{math}
    \end{center}
    We shall also refer to indexed variants of these; for example
    \begin{center}
        \begin{math}
            Split_n(\Gamma_1 ,...,\Gamma_n):\interp{\Gamma_1 ,...,\Gamma_n} \rightarrow \interp{\Gamma_1} \bullet ... \bullet \interp{\Gamma_n}
        \end{math}
    \end{center}
    which is defined in the obvious way.\\
    \indent The semantics of a term in context is then specified by a structural induction on the term.
    \begin{center}
        \begin{math}
            \begin{array}{rcl}
                \interp{x:\sigma \rhd x:\sigma} & \myeq & id_\sigma\\\\
                \interp{\Gamma_1,...,\Gamma_n \rhd f(M_1,...,M_n):\tau} & \myeq & Split_n(\Gamma_1,...,\Gamma_n);\interp{\Gamma_1 \rhd M_1:\sigma_1}
                \bullet ... \bullet \interp{\Gamma_n \rhd M_1:\sigma_n0};\interp{f}
            \end{array}
        \end{math}
    \end{center}
\end{definition}