\documentclass[11pt]{article}

\usepackage{amsmath,amssymb,amsthm}
\usepackage{hyperref}
\usepackage{mathpartir}
\usepackage[barr]{xy}
\usepackage{stmaryrd}
\usepackage{supertabular}
\usepackage{geometry}
\usepackage{fullpage}
\usepackage{mdframed}

\newtheorem{theorem}{Theorem}
\newtheorem{lemma}{Lemma}
\newtheorem{definition}{Definition}
\newtheorem{example}{Example}

\newcommand{\interp}[1]{[\negthinspace[#1]\negthinspace]}
\newcommand\myeq{\mathrel{\overset{\makebox[0pt]{\mbox{\normalfont\tiny\sffamily def}}}{=}}}

\title{Monoidal-Annex}
\author{Preston Keel \and Dr. Harley Eades}
\begin{document}
\maketitle
\tableofcontents

\section{Monoidal Categories}
\label{sec:monoidal_categories}
\begin{definition}
    A monoidal category $(\mathbb{C},\otimes,I,\alpha,\lambda,\rho)$ is a category $\mathbb{C}$, a bifunctor $\otimes : \mathbb{C} \times \mathbb{C} \rightarrow \mathbb{C}$, 
    an object $I \in \mathbb{C}$, and three natural isomorphisms $\alpha,\lambda,\rho$.  Where,
    \begin{center}
        \begin{math}
            \alpha = \alpha_{A,B,C}:A \otimes(B \otimes C) \cong (A \otimes B) \otimes C
        \end{math}
    \end{center}
    is natural for all $A,B,C \in \mathbb{C}$ and the diagram
    \begin{center}
        \begin{math}
            \bfig
            \ptriangle|all|/->`->`/<1500,500>[
                A \otimes (B \otimes (C \otimes D))`
                (A \otimes B) \otimes (C \otimes D)`
                A \otimes ((B \otimes C)\otimes D);
                \alpha`
                1 \otimes \alpha`
            ]
            \qtriangle(1500,0)|alr|/->``<-/<1500,500>[
                (A \otimes B) \otimes (C \otimes D)`
                ((A \otimes B)\otimes C)\otimes D`
                (A \otimes (B \otimes C))\otimes D;
                \alpha`
                `
                \alpha \otimes 1
                ]
            \morphism|b|/->/<3000,0>[
                A \otimes ((B \otimes C)\otimes D)`
                (A \otimes(B \otimes C))\otimes D;
                \alpha
                ]
            \efig
        \end{math}
    \end{center}
    commutes for all $A,B,C,D \in \mathbb{C}$.  $\gamma$ and $\rho$ are natural
    \begin{center}
        \begin{math}
            \begin{array}{lr}
                \gamma_A : I \otimes A \cong A & \rho_A : A \otimes I \cong A
            \end{array}
        \end{math}
    \end{center} 
    for all objects $A \in \mathbb{C}$, the diagram
    \begin{center}
        \begin{math}
            \bfig
                    \square|alrr|/->`->`->`=/<1000,500>[
                        A \otimes (I \otimes C)`
                        (A \otimes I) \otimes C`
                        A \otimes C`
                        A \otimes C;
                        \alpha`
                        1 \otimes \lambda`
                        \rho_A \otimes 1`]
                    \efig
        \end{math}
    \end{center}
    \cite{maclane1971}
\end{definition}


\begin{definition}
\label{def:monoidal_functor}
    A monoidal functor or lax monoidal functor $(F,m)$ between monoidal categories $(\mathbb{C}, \otimes, I)$ and
    $(\mathbb{D}, \otimes ', I')$ is a functor $F:\mathbb{C} \rightarrow \mathbb{D}$ equipped with nathral transformations
    \begin{center}
        \begin{math}
            \begin{array}{lr}
                m_{A,B}' : FA \otimes ' FB \rightarrow F(A \otimes B) & m'' : I' \rightarrow FI
            \end{array}
        \end{math}
    \end{center}
    where the following diagrams commute in the category $\mathbb{D}$ for all objects $A,B,C \in \mathbb{C}$
    \begin{center}
        \begin{math}
            \begin{array}{c}
                \bfig
                    \square|alrr|/->`->`->`/<1500,500>[
                        (FA \otimes ' FB) \otimes ' FC`
                        FA \otimes ' (FB \otimes ' FC)`
                        F(A \otimes B) \otimes ' FC`
                        FA \otimes ' F(B \otimes C);
                        \alpha '`
                        m \otimes FC`
                        FA \otimes m`]
                    \square(0,-500)|alrb|/`->`->`->/<1500,500>[
                        F(A \otimes B) \otimes ' FC`
                        FA \otimes ' F(B \otimes C)`
                        F((A \otimes B)\otimes C)`
                        F(A \otimes (B \otimes C));
                        `
                        m`
                        m`
                        F \alpha
                    ]
                    
                \efig
                \\\\
                \begin{array}{lr}
                    \bfig
                        \square|alra|/->`->`<-`->/<1000,500>[
                            FA \otimes ' I'`
                            FA`
                            FA \otimes ' FI`
                            F(A \otimes I);
                            \rho'`
                            FA \otimes ' m`
                            F\rho`
                            m
                        ]
                    \efig
                    &
                    \bfig
                        \square|alra|/->`->`<-`->/<1000,500>[
                            I' \otimes ' FB`
                            FB`
                            FI \otimes FB`
                            F(I \otimes B);
                            \gamma '`
                            m \otimes ' FB`
                            F\gamma`
                            m
                        ]
                    \efig
                \end{array}
            \end{array}
        \end{math}
    \end{center}
    \cite{mellies2009}
\end{definition}


\begin{definition}
\label{def:oplax_monoidal_functor}
    An oplax(colax/comonoidal) monoidal functor $(F,n)$ between monoidal categories $(\mathbb{C}, \otimes, I)$ and
    $(\mathbb{D}, \otimes ', I')$ is a functor $F:\mathbb{C} \rightarrow \mathbb{D}$ and natural transformations
    \begin{center}
        \begin{math}
            \begin{array}{lr}
                n_{A,B}' : F(A \otimes B) \rightarrow FA \otimes ' FB & n'' : FI \rightarrow I'
            \end{array}
        \end{math}
    \end{center}
    where the following diagrams commute in the category $\mathbb{D}$, for all objects $A,B,C \in \mathbb{C}$
    \begin{center}
        \begin{math}
            \begin{array}{c}
                \bfig
                \square|alrr|/->`->`->`/<1500,500>[
                    F((A \otimes B)\otimes C)`
                    F(A \otimes (B \otimes C))`
                    F(A \otimes B) \otimes ' FC`
                    FA \otimes ' F(B \otimes C);
                    F \alpha`
                    n`
                    n`
                ]
                \square(0,-500)|alrb|/`->`->`->/<1500,500>[
                    F(A \otimes B) \otimes ' FC`
                    FA \otimes ' F(B \otimes C)`
                    (FA \otimes ' FB) \otimes ' FC`
                    FA \otimes ' (FB \otimes ' FC);
                    `
                    n \otimes ' FC`
                    FA \otimes ' n`
                    \alpha '
                ]
                \efig
                \\\\
                \begin{array}{lr}
                    \bfig
                        \square|alra|/->`->`<-`->/<1000,500>[
                            F(A \otimes I)`
                            FA`
                            FA \otimes ' FI`
                            FA \otimes ' I';
                            F\rho`
                            n`
                            \rho '`
                            FA \otimes ' n
                        ]
                    \efig
                    &
                    \bfig
                        \square|alra|/->`->`<-`->/<1000,500>[
                            F(I \otimes B)`
                            FB`
                            FI \otimes ' FB`
                            I' \otimes ' FB;
                            F\gamma`
                            n`
                            \gamma '`
                            n \otimes ' FB
                        ]
                    \efig
                \end{array}
            \end{array}
        \end{math}
    \end{center}
    \cite{mellies2009}
\end{definition}



\begin{definition}
\label{def:monoidal_natural_transformation}
$(F,m)$ and $(G,n)$ are (lax) monoidal functors between the monoidal categories:
\begin{center}
    \begin{math}
        (\mathbb{C}, \otimes, I) \rightarrow (\mathbb{D}, \otimes ', I')
    \end{math}
\end{center}
A monoidal natural transformation
\begin{center}
    \begin{math}
        \theta : (F,m) \Rightarrow (G,n) : (\mathbb{C}, \otimes, I) \rightarrow (\mathbb{D}, \otimes ', I')
    \end{math}
\end{center}
between the monoidal functors $(F,m)$ and $(G,n)$ is a naural transformation
\begin{center}
    \begin{math}
        \theta : F \Rightarrow G : \mathbb{C} \rightarrow \mathbb{D}
    \end{math}
\end{center}
between the underlying functors, where the following diamgrams commute, for all objects $A,B \in \mathbb{C}$
\begin{center}
    \begin{math}
        \begin{array}{lr}
            \bfig
                \square|alrb|<1000,500>[
                    FA \otimes ' FB`
                    GA \otimes ' GB`
                    F(A \otimes B)`
                    G(A \otimes B);
                    \theta_A \otimes ' \theta_B`
                    m`
                    n`
                    \theta_{A \otimes B}
                ]
            \efig
            &
            \bfig
                \Atriangle<500,600>[
                    I'`
                    FI`
                    GI;
                    m`
                    n`
                    \theta_I
                ]
            \efig
        \end{array}
    \end{math}
\end{center}
\cite{mellies2009}
\end{definition}


\begin{definition}
\label{def:oplax_monoidal_natural_transformation}
    A monoidal natural transformation
    \begin{center}
        \begin{math}
            \theta : (F,m) \Rightarrow (G,n) : (\mathbb{C}, \otimes, I) \rightarrow (\mathbb{D}, \otimes ', I')
        \end{math}
    \end{center}
    between two oplax(colax/comonoidal) monoidal functors $(F,m)$ and $(G,n)$ is a natural transformation
    \begin{center}
        \begin{math}
            \theta : F \Rightarrow G : \mathbb{C} \rightarrow \mathbb{D}
        \end{math}
    \end{center}
    between the underlying functors, where the following diagrmas commute, for all objects $A,B \in \mathbb{C}$
    \begin{center}
        \begin{math}
            \begin{array}{rl}
                \bfig
                    \square<1000,500>[
                        F(A \otimes B)`
                        G(A \otimes B)`
                        FA \otimes ' FB`
                        GA \otimes ' GB;
                        \theta_{A \otimes B}`
                        m`
                        n`
                        \theta_A \otimes ' \theta_B
                    ]
                \efig
                &
                \bfig
                    \Vtriangle<500,600>[
                        FI`
                        GI`
                        I';
                        \theta_I`
                        m`
                        n
                    ]
                \efig
            \end{array}
        \end{math}
    \end{center}
    \cite{mellies2009}
\end{definition}


\begin{definition}
\label{def:biclosed_monoidal_category}
    A monoidal category $\mathbb{C}$ is said to be biclosed if every $- \otimes Y$ has a right adjoint $[Y,-]$ and every 
    $X \otimes -$ has a right adjoint $\interp{X,-}$ \cite{kelly1982}
\end{definition}


\begin{definition}
\label{def:monoidal_adjunction}
    Given a pair of (lax) monoidal functors:
    \begin{center}
        \begin{math}
            \begin{array}{lr}
                (F_*,m) : (\mathbb{C}, \otimes, I) \rightarrow (\mathbb{D}, \otimes ' , I')
                &
                (F^*, n) : (\mathbb{D}, \otimes ', I') \rightarrow (\mathbb{C},\otimes,I).
            \end{array}
        \end{math}
    \end{center}
    A monoidal adjunction
    \begin{center}
        \begin{math}
            (F_*,m) \dashv (F^*, n)
        \end{math}
    \end{center}
    between the monoidal functors is defined as an adjunction $(F_*,F^*,\eta,\epsilon)$ between the underlying functors
    \begin{center}
        \begin{math}
            \begin{array}{lr}
                F_* : \mathbb{C} \rightarrow \mathbb{D}
                &
                F^* : \mathbb{D} \rightarrow \mathbb{C}
            \end{array}
        \end{math}
    \end{center}
    whose natural transformations 
    \begin{center}
        \begin{math}
            \begin{array}{lr}
                \eta : id_C \Rightarrow F^* \circ F_* & \epsilon : F_* \circ F^* \Rightarrow id_D
            \end{array}
        \end{math}
    \end{center}
    are monoidal \cite{mellies2009}
\end{definition}


\section{Symmetric Monoidal Categories}
\label{sec:symmetric_monoidal_categories}
\begin{definition}
    A symmetric monoidal catergory (SMC), $(\mathbb{C},\otimes,I,\alpha,\lambda,\rho,\gamma)$, is a category $\mathbb{C}$ 
    equippped with a bifunctor $\otimes:\mathbb{C}\times\mathbb{C}\rightarrow\mathbb{C}$ with a neutral element I and natural isomorphisms 
    $\alpha, \lambda, \rho,$ and $\gamma$:
    \begin{enumerate}
        \item $\alpha_{A,B,C} : A \otimes (B \otimes C) \xrightarrow{\sim} (A \otimes B) \otimes C$
        \item $\lambda_A : I \otimes A \xrightarrow{\sim} A$
        \item $\rho_A : A \otimes I \xrightarrow{\sim} A$
        \item $\gamma_{A,B} : A \otimes B \xrightarrow{\sim} B \otimes A$
    \end{enumerate}
    which make the following 'coherence' diagrams commute.
    \begin{center}
        \begin{math}
            \begin{array}{c}
                \bfig
                \ptriangle|all|/->`->`/<1500,500>[
                    A \otimes (B \otimes (C \otimes D))`
                    (A \otimes B) \otimes (C \otimes D)`
                    A \otimes ((B \otimes C)\otimes D);
                    \alpha_{A,B,C \otimes D}`
                    id_A \otimes \alpha_{B,C,D}`
                ]
                \qtriangle(1500,0)|alr|/->``<-/<1500,500>[
                    (A \otimes B) \otimes (C \otimes D)`
                    ((A \otimes B)\otimes C)\otimes D`
                    (A \otimes (B \otimes C))\otimes D;
                    \alpha_{A \otimes B,C,C}`
                    `
                    \alpha_{A,B,C} \otimes id_D
                    ]
                \morphism|b|/->/<3000,0>[
                    A \otimes ((B \otimes C)\otimes D)`
                    (A \otimes(B \otimes C))\otimes D;
                    \alpha_{A,B \otimes C,D}
                    ]
                \efig
                \\\\
                \bfig
                \square|alrb|/->`->``->/<1000,500>[
                    (A \otimes B) \otimes C`
                    A \otimes (B \otimes C)`
                    (B \otimes A) \otimes C`
                    B \otimes (A \otimes C);
                    \alpha_{A,B,C}`
                    \gamma_{A,B} \otimes id_C`
                    `
                    \alpha_{B,A,C}
                ]
                \square(1000,0)|arlb|/->``->`->/<1000,500>[
                    A \otimes (B \otimes C)`
                    (B \otimes C) \otimes A`
                    B \otimes (A \otimes C)`
                    B \otimes (C \otimes A);
                    \gamma_{A,B \otimes C}`
                    `
                    \alpha_{B,C,A}`
                    id_B \otimes \gamma_{A,C}
                ]
                \efig
                \\\\
                \begin{array}{lr}
                    \bfig
                    \square|alrr|/->`->`->`=/<1000,500>[
                        A \otimes (I \otimes B)`
                        (A \otimes I) \otimes B`
                        A \otimes B`
                        A \otimes B;
                        \alpha_{A,I,B}`
                        id_A \otimes \lambda_B`
                        \rho_A \otimes id_B`]
                    \efig
                    &
                    \bfig
                    \qtriangle|abr|/->`=`->/[
                        A \otimes B`
                        B \otimes A`
                        A \otimes B;
                        \gamma_{A,B}`
                        `
                        \gamma_{B,A}]
                    \efig                    
                \end{array}\\\\
                \bfig
                \square|alrb|/->`->`->`=/[
                    A \otimes I`
                    I \otimes A`
                    A`
                    A;
                    \gamma_{A,I}`
                    \rho_A`
                    \lambda_A`
                    ]
                \efig
            \end{array}
        \end{math}
    \end{center}
    The following equality is also require to hold:
    \begin{center}
        \begin{math}
            \begin{array}{lcl}
                \lambda_I & = & \rho_I : I \otimes I \rightarrow I
            \end{array}
        \end{math}
    \end{center}
\end{definition}

\subsection{Symmetric Monoidal Closed Category}
\label{subsec:symmetric_monoidal_closded_category}
\begin{definition}
    A symmetric monoidal closed category (SMCC), $(\mathbb{C},\otimes,\multimap,I,\alpha,\lambda,\rho,\gamma)$,
    is a SMC such that for all objects A in $\mathbb{C}$, the functor $-\otimes A$ has a specified right adjoint $A\multimap -$.
\end{definition}

\section{Monoidal Double Category}
\label{sec:monoidal_double_category}
\begin{definition}
\label{def:monoidal_double_category}
    The strict double category $\mathbb{D}$ has a full double subcategory $\mathbb{M}$ of monoidal categoreis.
    These are viewd as:
    \begin{itemize}
        \item vertical double categories on a formal object $*$
        \item vertical arrows $A: * \rightarrow *$
        \item cells $a:A \rightarrow A'$
    \end{itemize}
    The horizontal arrows of $\mathbb{M}$ are monoidal functors(lax with respect to tensor product) and the vertical arrows 
    are comonoidal functors(colax). A cell $\alpha:(F RS G)$ associates to every object A in $\mathbb{A}$ with an arrow
    $\alpha A: GRA \rightarrow SFA$ in $\mathbb{D}$ which satisfies the naturality condition
    \begin{center}
        \begin{math}
            \begin{array}{lr}
                (GRa | \alpha v) = (\alpha u | SFa) & (\text{for } a:(u fg v) \text{ in } \mathbb{A})
            \end{array}
        \end{math}
    \end{center}
    and the coherence conditions
    \begin{center}
        \begin{math}
            \begin{array}{lr}
                (C\rho A | \alpha I | S \varphi A) = (\gamma RA | I | \sigma FA) & (\text{for a in }\mathbb{A})\\\\
                (G\rho (u,v) | \alpha w | S \varphi (u,v)) = (\gamma (Ru,Rv) | (\alpha u \otimes \alpha v) | \sigma (Fu,Fv)) & (\text{for }w = u\otimes v \text{ in } \mathbb{A})
            \end{array}
        \end{math}
    \end{center}
    and have the following diagrams
    \begin{center}
        \begin{math}
            \begin{array}{cc}
                \bfig
                    \square<500,300>[
                        GRA`
                        SFA`
                        GRA'`
                        SFA';
                        \alpha A`
                        GRa`
                        SFa`
                        \alpha A'
                    ]
                \efig
                & 
                \bfig
                    \Atriangle|lrb|/<-``->/<300,300>[
                        GRI`
                        GI`
                        I;
                        G\rho`
                        `
                        \gamma
                    ]
                    \Atriangle(600,0)|lrb|/`->`->/<300,300>[
                        SFI`
                        I`
                        SI;
                        `
                        S\varphi`
                        \sigma
                    ]
                    \morphism(300,300)|a|<600,0>[
                        GRI`
                        SFI;
                        \alpha I
                    ]
                \efig
                \\\\
                \begin{array}{c}
                    \bfig
                        \square|alrb|/->`<-`->`/<1000,500>[
                            GR(A \otimes A')`
                            SF(A \otimes A')`
                            G(RA \otimes RA')`
                            S(FA \otimes FA');
                            \alpha(A \otimes A')`
                            G\rho`
                            S\varphi`
                        ]
                        \square(0,-500)|alrb|/`->`<-`->/<1000,500>[
                            G(RA \otimes RA')`
                            S(FA \otimes FA')`
                            GRA \otimes GRA'`
                            SFA \otimes SFA';
                            `
                            \gamma R`
                            \sigma F`
                            \alpha A \otimes \alpha A'
                        ]
                    \efig
                \end{array}
            \end{array}
        \end{math}
    \end{center}
    where the lax monoidal functor R has comparison arrows 
    \begin{center}
        \begin{math}
            \begin{array}{l}
                \rho = \rho (*): I \rightarrow RI
                \\
                \rho(A, A'): RA \otimes RA' \rightarrow R(A \otimes A')
            \end{array}
        \end{math}
    \end{center}
    \cite{grandis2004}
\end{definition}

\section{The Double Category of (co)Lax Monoidal Functors}
\label{sec:the_double_category_of_colax_monoidal_functors}
\newcommand{\cat}[1]{\mathcal{#1}}
\newcommand{\mto}{\to}

\begin{definition}
  \label{def:lax-monoidal-multifunctors}
  A \emph{lax monoidal multifunctor} from $(\cat{C}_1,\ldots,\cat{C}_n)$ to $\cat{D}$ is a functor:
  \[
  F : \cat{C}_1 \times \cdots \times \cat{C}_n \mto \cat{D}
  \]
  such that:
  \begin{itemize}
  \item Each functor,
    $(F^i, m^i(\overrightarrow{A_1},-,\overrightarrow{A_{i+1}}), m^i_{I_i}(\overrightarrow{A_1},-,\overrightarrow{A_{i+1}})$
    defined by
    $F^i(X) = F(\overrightarrow{A_1},X,\overrightarrow{A_{i+1}}) : \cat{C}_i \mto \cat{D}$,
    for $1 \leq i \leq n$ is lax monoidal.
  \item The following equations hold for any $1 \leq i < j \leq n$:
    \begin{itemize}
    \item $m_{I_i}(\overrightarrow{I_1},-,\overrightarrow{I_{i+1}}) = m_{I_j}(\overrightarrow{I_1},-,\overrightarrow{I_{i+1}})$,
    \item
      \begin{math}
        \begin{array}{lll}
          \\
          (m^i_{I_i}(\overrightarrow{A_1},-,\overrightarrow{A_{i+1}},X,\overrightarrow{A_{j+1}}) \otimes m^i_{I_i}(\overrightarrow{A_1},-,\overrightarrow{A_{i+1}},Y,\overrightarrow{A_{j+1}}));m^j_{X,Y}(\overrightarrow{A_1},I_i,\overrightarrow{A_{i+1}},-,\overrightarrow{A_{j+1}}) \\ = \iota;m^{i}_{I_i}(\overrightarrow{A_1},-,\overrightarrow{A_{i+1}},X \otimes Y,\overrightarrow{A_{j+1}})
        \end{array}
      \end{math}

    \item \begin{math}
        \begin{array}{lll}
          \\
          (m^j_{I_j}(\overrightarrow{A_1},X,\overrightarrow{A_{i+1}},-,\overrightarrow{A_{j+1}}) \otimes m^j_{I_j}(\overrightarrow{A_1},Y,\overrightarrow{A_{i+1}},-,\overrightarrow{A_{j+1}}));m^i_{X,Y}(\overrightarrow{A_1},-,\overrightarrow{A_{i+1}},I_j,\overrightarrow{A_{j+1}}) \\ = \iota;m^{j}_{I_j}(\overrightarrow{A_1},X \otimes Y,\overrightarrow{A_{i+1}},-,\overrightarrow{A_{j+1}})
        \end{array}
    \end{math}
    \end{itemize}    
  \end{itemize}
\end{definition}
\noindent
In the above definition $\iota$ is the isomorphism $\iota : I \otimes I \mto I$ which holds in any monoidal category.

\begin{definition}
  \label{def:lax-monoidal-multifunctors}
  A \emph{symmetric lax monoidal multifunctor} from $(\cat{C}_1,\ldots,\cat{C}_n)$ to $\cat{D}$ is a lax monoidal multifunctor
  \[F : (\cat{C}_1,\ldots,\cat{C}_n) \mto \cat{D}\]
  such that:
  \begin{itemize}
  \item Each functor,
    $(F^i, m^i(\overrightarrow{A_1},-,\overrightarrow{A_{i+1}}), m^i_{I_i}(\overrightarrow{A_1},-,\overrightarrow{A_{i+1}})$
    defined by
    $F^i(X) = F(\overrightarrow{A_1},X,\overrightarrow{A_{i+1}}) : \cat{C}_i \mto \cat{D}$,
    for $1 \leq i \leq n$ is symmetric lax monoidal.
  \item The following additional coherence axiom holds for any $1 \leq i < j \leq n$:
    \begin{itemize}
      \vspace{-10px}
    \item \begin{math}
      \begin{array}{lll}
        \\
        (m^i_{X,Y}(\overrightarrow{A_1},-,\overrightarrow{A_{i+1}},P,\overrightarrow{A_{j+1}}) \otimes m^i_{X,Y}(\overrightarrow{A_1},-,\overrightarrow{A_{i+1}},Q,\overrightarrow{A_{j+1}}));m^j_{P,Q}(\overrightarrow{A_1},X \otimes Y,\overrightarrow{A_{i+1}},-,\overrightarrow{A_{j+1}}) \\
        = \tau;(m^j_{P,Q}(\overrightarrow{A_1},X,\overrightarrow{A_{i+1}},-,\overrightarrow{A_{j+1}}) \otimes m^j_{P,Q}(\overrightarrow{A_1},Y,\overrightarrow{A_{i+1}},-,\overrightarrow{A_{j+1}}));m^{i}_{X,Y}(\overrightarrow{A_1},-,\overrightarrow{A_{i+1}},P \otimes Q,\overrightarrow{A_{j+1}})
      \end{array}
    \end{math}
    \end{itemize}
  \end{itemize}
\end{definition}
\noindent
In the above definition $\tau$ is the isomorphism
$\tau : (A \otimes B) \otimes (C \otimes D) \mto (A \otimes C) \otimes (B \otimes D)$
which holds in any monoidal category.


% section the_double_category_of_(co)lax_monoidal_functors (end)

\section{Monoidal Monads}
\label{sec:monoidal_monads}
\begin{definition}
\label{def:monoidal_monad}
    Let $\mathbb{C}=(C,\otimes,I)$ be a monoidal category.  A monoidal monad $\mathbb{T} = (T,\mu,\eta)$ is a monad in the 2-category of:
    \begin{enumerate}
        \item monoidal categories
        \item monoidal functors
        \item monoidal transformations
    \end{enumerate}
    where the functor $T:C \rightarrow C$ comes with maps $\kappa = (\kappa_{X,Y}: TX \otimes TY \rightarrow T(X \otimes Y))_{X,Y \in ob\, C}$
    that make $(T,\kappa,\eta_I)$ a monoidal functor and $\mu,\eta$ monoidal natural transformations \cite{seal2013}
\end{definition}

\bibliography{references}
\bibliographystyle{plain}
\end{document}
