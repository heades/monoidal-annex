\documentclass[11pt]{article}

\usepackage{amsmath,amssymb,amsthm}
\usepackage{hyperref}
\usepackage{mathpartir}
\usepackage[barr]{xy}
\usepackage{stmaryrd}
\usepackage{supertabular}
\usepackage{geometry}
\usepackage{fullpage}
\usepackage{mdframed}

\newtheorem{theorem}{Theorem}
\newtheorem{lemma}{Lemma}
\newtheorem{definition}{Definition}
\newtheorem{example}{Example}

\newcommand{\interp}[1]{[\negthinspace[#1]\negthinspace]}
\newcommand\myeq{\mathrel{\overset{\makebox[0pt]{\mbox{\normalfont\tiny\sffamily def}}}{=}}}

\title{Monoidal-Annex}
\author{Preston Keel \and Dr. Harley Eades}
\begin{document}
\maketitle
\tableofcontents

\section{Monoidal Categories}
\label{sec:monoidal_categories}
\subsection{Symmetric Monoidal Category}
\label{subsec:symmetric_monoidal_category}
\begin{definition}
    A symmetric monoidal catergory (SMC), $(\mathbb{C},\bullet,1,\alpha,\lambda,\rho,\gamma)$, is a category $\mathbb{C}$ 
    equippped with a bifunctor $\bullet:\mathbb{C}\times\mathbb{C}\rightarrow\mathbb{C}$ with a neutral element 1 and natural isomorphisms 
    $\alpha, \lambda, \rho,$ and $\gamma$:
    \begin{enumerate}
        \item $\alpha_{A,B,C} : A \bullet (B \bullet C) \xrightarrow{\sim} (A \bullet B) \bullet C$
        \item $\lambda_A : 1 \bullet A \xrightarrow{\sim} A$
        \item $\rho_A : A \bullet 1 \xrightarrow{\sim} A$
        \item $\gamma_{A,B} : A \bullet B \xrightarrow{\sim} B \bullet A$
    \end{enumerate}
    which make the following 'coherence' diagrams commute.
    \begin{center}
        \begin{math}
            \begin{array}{c}
                \bfig
                \ptriangle|all|/->`->`/<1500,500>[
                    A \bullet (B \bullet (C \bullet D))`
                    (A \bullet B) \bullet (C \bullet D)`
                    A \bullet ((B \bullet C)\bullet D);
                    \alpha_{A,B,C \bullet D}`
                    id_A \bullet \alpha_{B,C,D}`
                ]
                \qtriangle(1500,0)|alr|/->``<-/<1500,500>[
                    (A \bullet B) \bullet (C \bullet D)`
                    ((A \bullet B)\bullet C)\bullet D`
                    (A \bullet (B \bullet C))\bullet D;
                    \alpha_{A \bullet B,C,C}`
                    `
                    \alpha_{A,B,C} \bullet id_D
                    ]
                \morphism|b|/->/<3000,0>[
                    A \bullet ((B \bullet C)\bullet D)`
                    (A \bullet(B \bullet C))\bullet D;
                    \alpha_{A,B \bullet C,D}
                    ]
                \efig
                \\\\
                \bfig
                \square|alrb|/->`->``->/<1000,500>[
                    (A \bullet B) \bullet C`
                    A \bullet (B \bullet C)`
                    (B \bullet A) \bullet C`
                    B \bullet (A \bullet C);
                    \alpha_{A,B,C}`
                    \gamma_{A,B} \bullet id_C`
                    `
                    \alpha_{B,A,C}
                ]
                \square(1000,0)|arlb|/->``->`->/<1000,500>[
                    A \bullet (B \bullet C)`
                    (B \bullet C) \bullet A`
                    B \bullet (A \bullet C)`
                    B \bullet (C \bullet A);
                    \gamma_{A,B \bullet C}`
                    `
                    \alpha_{B,C,A}`
                    id_B \bullet \gamma_{A,C}
                ]
                \efig
                \\\\
                \begin{array}{lr}
                    \bfig
                    \square|alrr|/->`->`->`=/<1000,500>[
                        A \bullet (1 \bullet B)`
                        (A \bullet 1) \bullet B`
                        A \bullet B`
                        A \bullet B;
                        \alpha_{A,1,B}`
                        id_A \bullet \lambda_B`
                        \rho_A \bullet id_B`]
                    \efig
                    &
                    \bfig
                    \qtriangle|abr|/->`=`->/[
                        A \bullet B`
                        B \bullet A`
                        A \bullet B;
                        \gamma_{A,B}`
                        `
                        \gamma_{B,A}]
                    \efig                    
                \end{array}\\\\
                \bfig
                \square|alrb|/->`->`->`=/[
                    A \bullet 1`
                    1 \bullet A`
                    A`
                    A;
                    \gamma_{A,1}`
                    \rho_A`
                    \lambda_A`
                    ]
                \efig
            \end{array}
        \end{math}
    \end{center}
    The following equality is also require to hold:
    \begin{center}
        \begin{math}
            \begin{array}{lcl}
                \lambda_1 & = & \rho_1 : 1 \bullet 1 \rightarrow 1
            \end{array}
        \end{math}
    \end{center}
\end{definition}

\subsection{Symmetric Monoidal Closed Category}
\label{subsec:symmetric_monoidal_closded_category}
\begin{definition}
    A symmetric monoidal closed category (SMCC), $(\mathbb{C},\bullet,\multimap,1,\alpha,\lambda,\rho,\gamma)$,
    is a SMC such that for all objects A in $\mathbb{C}$, the functor $-\otimes A$ has a specified right adjoint $A\multimap -$.
    \\
    \\
    \indent Let $\mathbb{C}$ be a SMC $(\mathbb{C},\bullet,1,\alpha,\lambda,\rho,\gamma)$. A structure , $M$ , in $\mathbb{C}$ for a given signature 
    $S_g$ is specified by giving an object $\interp{\sigma}$ in $\mathbb{C}$ for each type $\sigma$, and a morphism 
    $\interp{f}:\interp{\sigma_1}\bullet...\bullet \interp{\sigma_n} \rightarrow \interp{\tau}$ in $\mathbb{C}$ for each function symbol
    $f:\sigma_1 , ... , \sigma_n \rightarrow \tau$.  In the case where $n = 0$ then the structure assigns a morphism 
    $\interp{c}:1 \rightarrow \interp{\tau}$ to a constant $c:\tau$.
    \\
    \indent Given a context $\Gamma = [x_1:\sigma_1,...,x_n:\sigma_n]$ we define $\interp{\Gamma}$ to be the product
    $\interp{\sigma_1}\bullet ... \bullet \interp{\sigma_n}$.  We represent the empyt context with the neutral element 1.  We need
    to define the bracketing convention.  It shall be assumed that the tensor product is left associative, i.e. $A_1 \bullet A_2\bullet ... \bullet A_n$ 
    will be taken to mean $(...(A_1 \bullet A_2)\bullet...)\bullet A_n$.  We find it useful to define two 'book-keeping' functions,
    \begin{center}
        \begin{math}
            \begin{array}{l}
                Split(\Gamma,\Delta):\interp{\Gamma,\Delta} \rightarrow \interp{\Gamma} \bullet \interp{\Delta}\\\\
                Split(\Gamma,\Delta) \myeq
                \begin{cases}
                    \lambda^{-1}_{\Delta} & \text{If }\Gamma \text{ empty}\\
                    \rho^{-1}_{\Gamma} & \text{If } \Delta \text{ empty}\\
                    id_{\Gamma \bullet A} & \text{If } \Delta = A\\
                    Split(\Gamma,\Delta')\bullet id_{A};\alpha^{-1}_{\Gamma,\Delta',A} & \text{If } \Delta = \Delta',A
                \end{cases}
                \\\\
                Join(\Gamma,\Delta): \interp{\Gamma,\Delta} \rightarrow \interp{\Gamma} \bullet \interp{\Delta}\\\\
                Join(\Gamma,\Delta) \myeq 
                \begin{cases}
                    \lambda_\Delta & \text{If } \Gamma \text{ empty}\\
                    \rho_\Gamma & \text{If } \Delta \text{ empty}\\
                    id_{\Gamma \bullet A} & \text{If }\Delta = A\\
                    \alpha_{\Gamma,\Delta',A};Join(\Gamma,\Delta')\bullet id_A & \text{If } \Delta = \Delta', A
                \end{cases}
            \end{array}
        \end{math}
    \end{center}
    We shall also refer to indexed variants of these; for example
    \begin{center}
        \begin{math}
            Split_n(\Gamma_1 ,...,\Gamma_n):\interp{\Gamma_1 ,...,\Gamma_n} \rightarrow \interp{\Gamma_1} \bullet ... \bullet \interp{\Gamma_n}
        \end{math}
    \end{center}
    which is defined in the obvious way.\\
    \indent The semantics of a term in context is then specified by a structural induction on the term.
    \begin{center}
        \begin{math}
            \begin{array}{rcl}
                \interp{x:\sigma \rhd x:\sigma} & \myeq & id_\sigma\\\\
                \interp{\Gamma_1,...,\Gamma_n \rhd f(M_1,...,M_n):\tau} & \myeq & Split_n(\Gamma_1,...,\Gamma_n);\interp{\Gamma_1 \rhd M_1:\sigma_1}
                \bullet ... \bullet \interp{\Gamma_n \rhd M_1:\sigma_n0};\interp{f}
            \end{array}
        \end{math}
    \end{center}
\end{definition}

\section{Symmetric Monoidal Categories}
\label{sec:symmetric_monoidal_categories}
\begin{definition}
    \label{def:symmetric_monoidal_category}
    A symmetric monoidal catergory (SMC), $(\mathbb{C},\otimes,I,\alpha,\lambda,\rho,\gamma)$, is a category $\mathbb{C}$ 
    equippped with a bifunctor $\otimes:\mathbb{C}\times\mathbb{C}\rightarrow\mathbb{C}$ with a neutral element I and natural isomorphisms 
    $\alpha, \lambda, \rho,$ and $\gamma$:
    \begin{enumerate}
        \item $\alpha_{A,B,C} : A \otimes (B \otimes C) \xrightarrow{\sim} (A \otimes B) \otimes C$
        \item $\lambda_A : I \otimes A \xrightarrow{\sim} A$
        \item $\rho_A : A \otimes I \xrightarrow{\sim} A$
        \item $\gamma_{A,B} : A \otimes B \xrightarrow{\sim} B \otimes A$
    \end{enumerate}
    which make the following 'coherence' diagrams commute.
    \begin{center}
        \begin{math}
            \begin{array}{c}
                \bfig
                \ptriangle|all|/->`->`/<1500,500>[
                    A \otimes (B \otimes (C \otimes D))`
                    (A \otimes B) \otimes (C \otimes D)`
                    A \otimes ((B \otimes C)\otimes D);
                    \alpha_{A,B,C \otimes D}`
                    id_A \otimes \alpha_{B,C,D}`
                ]
                \qtriangle(1500,0)|alr|/->``<-/<1500,500>[
                    (A \otimes B) \otimes (C \otimes D)`
                    ((A \otimes B)\otimes C)\otimes D`
                    (A \otimes (B \otimes C))\otimes D;
                    \alpha_{A \otimes B,C,C}`
                    `
                    \alpha_{A,B,C} \otimes id_D
                    ]
                \morphism|b|/->/<3000,0>[
                    A \otimes ((B \otimes C)\otimes D)`
                    (A \otimes(B \otimes C))\otimes D;
                    \alpha_{A,B \otimes C,D}
                    ]
                \efig
                \\\\
                \bfig
                \square|alrb|/->`->``->/<1000,500>[
                    (A \otimes B) \otimes C`
                    A \otimes (B \otimes C)`
                    (B \otimes A) \otimes C`
                    B \otimes (A \otimes C);
                    \alpha_{A,B,C}`
                    \gamma_{A,B} \otimes id_C`
                    `
                    \alpha_{B,A,C}
                ]
                \square(1000,0)|arlb|/->``->`->/<1000,500>[
                    A \otimes (B \otimes C)`
                    (B \otimes C) \otimes A`
                    B \otimes (A \otimes C)`
                    B \otimes (C \otimes A);
                    \gamma_{A,B \otimes C}`
                    `
                    \alpha_{B,C,A}`
                    id_B \otimes \gamma_{A,C}
                ]
                \efig
                \\\\
                \begin{array}{lr}
                    \bfig
                    \square|alrr|/->`->`->`=/<1000,500>[
                        A \otimes (I \otimes B)`
                        (A \otimes I) \otimes B`
                        A \otimes B`
                        A \otimes B;
                        \alpha_{A,I,B}`
                        id_A \otimes \lambda_B`
                        \rho_A \otimes id_B`]
                    \efig
                    &
                    \bfig
                    \qtriangle|abr|/->`=`->/[
                        A \otimes B`
                        B \otimes A`
                        A \otimes B;
                        \gamma_{A,B}`
                        `
                        \gamma_{B,A}]
                    \efig                    
                \end{array}\\\\
                \bfig
                \square|alrb|/->`->`->`=/[
                    A \otimes I`
                    I \otimes A`
                    A`
                    A;
                    \gamma_{A,I}`
                    \rho_A`
                    \lambda_A`
                    ]
                \efig
            \end{array}
        \end{math}
    \end{center}
    The following equality is also require to hold:
    \begin{center}
        \begin{math}
            \begin{array}{lcl}
                \lambda_I & = & \rho_I : I \otimes I \rightarrow I
            \end{array}
        \end{math}
    \end{center}
    \cite{bierman1993}
\end{definition}



\begin{definition}
\label{def:symmetric_monoidal_closded_category}
    A symmetric monoidal closed category (SMCC), $(\mathbb{C},\otimes,\multimap,I,\alpha,\lambda,\rho,\gamma)$,
    is a SMC such that for all objects A in $\mathbb{C}$, the functor $-\otimes A$ has a specified right adjoint $A\multimap -$.
\end{definition}


\section{Monoidal Double Category}
\label{sec:monoidal_double_category}
\begin{definition}
\label{def:monoidal_double_category}
    The strict double category $\mathbb{D}$ has a full double subcategory $\mathbb{M}$ of monoidal categoreis.
    These are viewd as:
    \begin{itemize}
        \item vertical double categories on a formal object $*$
        \item vertical arrows $A: * \rightarrow *$
        \item cells $a:A \rightarrow A'$
    \end{itemize}
    The horizontal arrows of $\mathbb{M}$ are monoidal functors(lax with respect to tensor product) and the vertical arrows 
    are comonoidal functors(colax). A cell $\alpha:(F RS G)$ associates to every object A in $\mathbb{A}$ with an arrow
    $\alpha A: GRA \rightarrow SFA$ in $\mathbb{D}$ which satisfies the naturality condition
    \begin{center}
        \begin{math}
            \begin{array}{lr}
                (GRa | \alpha v) = (\alpha u | SFa) & (\text{for } a:(u fg v) \text{ in } \mathbb{A})
            \end{array}
        \end{math}
    \end{center}
    and the coherence conditions
    \begin{center}
        \begin{math}
            \begin{array}{lr}
                (C\rho A | \alpha I | S \varphi A) = (\gamma RA | I | \sigma FA) & (\text{for a in }\mathbb{A})\\
                (G\rho (u,v) | \alpha w | S \varphi (u,v)) = (\gamma (Ru,Rv) | (\alpha u \otimes \alpha v) | \sigma (Fu,Fv)) & (\text{for }w = u\otimes v \text{ in } \mathbb{A})
            \end{array}
        \end{math}
    \end{center}
    and have the following diagrmas
    \begin{center}
        \begin{math}
            \begin{array}{lr}
                \bfig
                    \square<500,300>[
                        GRA`
                        SFA`
                        GRA'`
                        SFA';
                        \alpha A`
                        GRa`
                        SFa`
                        \alpha A'
                    ]
                \efig
                & 
                \bfig
                    \Atriangle|lrb|/<-``->/<300,300>[
                        GRI`
                        GI`
                        I;
                        G\rho`
                        `
                        \gamma
                    ]
                    \Atriangle(600,0)|lrb|/`->`->/<300,300>[
                        SFI`
                        I`
                        SI;
                        `
                        S\varphi`
                        \sigma
                    ]
                    \morphism(300,300)|a|<600,0>[
                        GRI`
                        SFI;
                        \alpha I
                    ]
                \efig
            \end{array}
        \end{math}
    \end{center}
\end{definition}

\section{The Double Category of (co)Lax Monoidal Functors}
\label{sec:the_double_category_of_colax_monoidal_functors}
\newcommand{\cat}[1]{\mathcal{#1}}
\newcommand{\mto}{\to}

\begin{definition}
  \label{def:lax-monoidal-multifunctors}
  A \emph{lax monoidal multifunctor} from $(\cat{C}_1,\ldots,\cat{C}_n)$ to $\cat{D}$ is a functor:
  \[
  F : \cat{C}_1 \times \cdots \times \cat{C}_n \mto \cat{D}
  \]
  such that:
  \begin{itemize}
  \item Each functor,
    $(F^i, m^i(\overrightarrow{A_1},-,\overrightarrow{A_{i+1}}), m^i_{I_i}(\overrightarrow{A_1},-,\overrightarrow{A_{i+1}})$
    defined by
    $F^i(X) = F(\overrightarrow{A_1},X,\overrightarrow{A_{i+1}}) : \cat{C}_i \mto \cat{D}$,
    for $1 \leq i \leq n$ is lax monoidal.
  \item The following equations hold for any $1 \leq i < j \leq n$:
    \begin{itemize}
    \item $m_{I_i}(\overrightarrow{I_1},-,\overrightarrow{I_{i+1}}) = m_{I_j}(\overrightarrow{I_1},-,\overrightarrow{I_{i+1}})$,
    \item
      \begin{math}
        \begin{array}{lll}
          \\
          (m^i_{I_i}(\overrightarrow{A_1},-,\overrightarrow{A_{i+1}},X,\overrightarrow{A_{j+1}}) \otimes m^i_{I_i}(\overrightarrow{A_1},-,\overrightarrow{A_{i+1}},Y,\overrightarrow{A_{j+1}}));m^j_{X,Y}(\overrightarrow{A_1},I_i,\overrightarrow{A_{i+1}},-,\overrightarrow{A_{j+1}}) \\ = \iota;m^{i}_{I_i}(\overrightarrow{A_1},-,\overrightarrow{A_{i+1}},X \otimes Y,\overrightarrow{A_{j+1}})
        \end{array}
      \end{math}

    \item \begin{math}
        \begin{array}{lll}
          \\
          (m^j_{I_j}(\overrightarrow{A_1},X,\overrightarrow{A_{i+1}},-,\overrightarrow{A_{j+1}}) \otimes m^j_{I_j}(\overrightarrow{A_1},Y,\overrightarrow{A_{i+1}},-,\overrightarrow{A_{j+1}}));m^i_{X,Y}(\overrightarrow{A_1},-,\overrightarrow{A_{i+1}},I_j,\overrightarrow{A_{j+1}}) \\ = \iota;m^{j}_{I_j}(\overrightarrow{A_1},X \otimes Y,\overrightarrow{A_{i+1}},-,\overrightarrow{A_{j+1}})
        \end{array}
    \end{math}
    \end{itemize}    
  \end{itemize}
\end{definition}
\noindent
In the above definition $\iota$ is the isomorphism $\iota : I \otimes I \mto I$ which holds in any monoidal category.

\begin{definition}
  \label{def:lax-monoidal-multifunctors}
  A \emph{symmetric lax monoidal multifunctor} from $(\cat{C}_1,\ldots,\cat{C}_n)$ to $\cat{D}$ is a lax monoidal multifunctor
  \[F : (\cat{C}_1,\ldots,\cat{C}_n) \mto \cat{D}\]
  such that:
  \begin{itemize}
  \item Each functor,
    $(F^i, m^i(\overrightarrow{A_1},-,\overrightarrow{A_{i+1}}), m^i_{I_i}(\overrightarrow{A_1},-,\overrightarrow{A_{i+1}})$
    defined by
    $F^i(X) = F(\overrightarrow{A_1},X,\overrightarrow{A_{i+1}}) : \cat{C}_i \mto \cat{D}$,
    for $1 \leq i \leq n$ is symmetric lax monoidal.
  \item The following additional coherence axiom holds for any $1 \leq i < j \leq n$:
    \begin{itemize}
      \vspace{-10px}
    \item \begin{math}
      \begin{array}{lll}
        \\
        (m^i_{X,Y}(\overrightarrow{A_1},-,\overrightarrow{A_{i+1}},P,\overrightarrow{A_{j+1}}) \otimes m^i_{X,Y}(\overrightarrow{A_1},-,\overrightarrow{A_{i+1}},Q,\overrightarrow{A_{j+1}}));m^j_{P,Q}(\overrightarrow{A_1},X \otimes Y,\overrightarrow{A_{i+1}},-,\overrightarrow{A_{j+1}}) \\
        = \tau;(m^j_{P,Q}(\overrightarrow{A_1},X,\overrightarrow{A_{i+1}},-,\overrightarrow{A_{j+1}}) \otimes m^j_{P,Q}(\overrightarrow{A_1},Y,\overrightarrow{A_{i+1}},-,\overrightarrow{A_{j+1}}));m^{i}_{X,Y}(\overrightarrow{A_1},-,\overrightarrow{A_{i+1}},P \otimes Q,\overrightarrow{A_{j+1}})
      \end{array}
    \end{math}
    \end{itemize}
  \end{itemize}
\end{definition}
\noindent
In the above definition $\tau$ is the isomorphism
$\tau : (A \otimes B) \otimes (C \otimes D) \mto (A \otimes C) \otimes (B \otimes D)$
which holds in any monoidal category.


% section the_double_category_of_(co)lax_monoidal_functors (end)


\bibliography{references}
\bibliographystyle{plain}
\end{document}
